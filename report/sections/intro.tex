%%% Local Variables:
%%% mode: latex
%%% TeX-master: "../report"
%%% End:

\subsection{Programming languages and compilers in general}
Programming languages are the means for programmers to express what the machine should compute, or in other words; what a program should do.
The two main divisions between programming languages are likely whether they are High- or Low-level and which `programming paradigm' they follow. These paradigms will be described briefly in the following section (\ref{sec:paradigms}).

Low-level languages are close to what computers `natively' understand, which is machine code/instructions, but are harder for humans to read and write. Examples hereof are FORTRAN and COBOL. Usually, high-level languages are used for faster development, better readability and ease of maintenance, among other things, although this comes at the cost of performance. Thus, for many purposes programmers prefer higher level languages. After being written, these languages have to be translated into something the machine can interpret, and this is done with a so-called compiler.

A compiler is a program that translates source code in a specific language into another, usually lower level, language. Thus, a programming language is in fact implemented through a compiler.


\subsection{Programming paradigms}
\label{sec:paradigms}
Through the history of computer science, many different approaches to the design and implementation of programming languages have been taken. Most languages can be categorized as being either in the imperative or declarative style. Code written in imperative style is more explicit in \emph{how} a problem is solved, whereas declarative languages tend to describe \emph{what} is to be computed. There exists multiple language paradigms that fall under one or both of these categories.

A very popular paradigm is \textbf{object-oriented} programming (including \code{C++} and \code{Java}), in which data and algorithms are structured into objects that contain methods that can change the state of data in objects.

Another imperative approach are \textbf{procedural} languages like \code{C}, which is somewhat simpler and is characterized by subroutines, scoped variables and some sort of modules.

A very different style is the highly declarative \textbf{logic} programming, in which execution order and control-flow does not exist, but programs consist of logic clauses that define relationships and implications.

The final paradigm we will describe is \textbf{functional} programming. This is another declarative approach, which is characterized by functions and their composition as the fundamental means of expressing computations. Often, programs are written in such a way that most functions are pure, meaning that they will \emph{always} return the same output given the same input. This is a direct result of not having side-effects in the language. A side-effect is when a function not only returns a value, but also alters some state, effecting the output. As this is looked at as one of functional programming's greatest features, it is also one of its greatest limitation, because it is hard to write a useful program without some sort of IO, e.g.\ writing to a file.

Another great advantage of functional programming is that, because of the transparent nature of its structure, it is arguably easier to implement functionality that can be executed concurrently by a processor. This has major performance advantages, as the technological development of processors has lately been focused on adding cores rather than increasing the clock speed of the individual cores. In fact, the clock speed of processors already stagnated around the year 2002~\cite{sutter09}.


\subsection{The problem}
The goal of this project is to gain knowledge in a field which we have little prior knowledge about, as we at the beginning of this project have not had any formal education within the subject. In short, we strive to get a good understanding of the concepts of functional programming and compilation of programming languages in general. We will achieve this by designing our own functional language named Kite, and implementing a compiler for it.


\subsection{The structure of the report}
The report is structured such that it somewhat follows the steps we have taken from idea to working product. We will start by describing the problem and purpose of the project (chapter \ref{sec:probanal}), and then to define what our requirements for completion are (chapter \ref{sec:requirements}). Next we will talk about the syntactic and semantic design of the Kite language (chapter \ref{sec:kite-design}) and how we architecturally designed the corresponding compiler (chapter \ref{sec:compiler-design}). To sum up our work, we give an in-depth explanation of the implementation of each of the involved modules and their purpose (chapter \ref{sec:impl}). Finally, we evaluate how the implementation turned out (chapter \ref{sec:evaluation}) and discuss what further development of Kite could lead to (chapter \ref{sec:discussion}).
