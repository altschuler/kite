%%% Local Variables:
%%% mode: latex
%%% TeX-master: "../report"
%%% End:

\subsection{Purpose of the project}
Our project is not concerned with solving a specific problem related to compilation of functional programming languages. Rather, it is a project in which we aim to gain a thourough understanding of modern compiler theory and implementation, as well as become proficient with functional development. We have made an effort to implement the compiler using modern techniques. That said, a compiler is a very complex piece of software and given our limited time, it is far beyond our scope to implement a language that could be used for production purposes. What we are striving to achieve, is to implement a working language with all the elements of a basic functional language, expressive syntax, simple optimizations and target language that we can execute.

There are two main aspects of the project. One is that of functional programming language theory and design, the other is that of compiler implementation and techniques. Even though they are very intertwined in the implementation, we will often touch on the two aspects separately.

Next, we will describe some properties of functional languages and compilers that we would like to implement in Kite. Further, we will introduce some of the notation used throughout the rest of the report.

\subsection{Properties of functional languages}
We will describe function types using the following notation: let $f$ be a function with a domain of $Int$ and codomain of $Float$. We denote this using '$:$', meaning ``of type'' or ``has type''.

\[ f: Int \to Float \]

Note that the concrete types are capitalized, and we use small letters to indicate type variables, also known as polymorphic types. For instance, $f: a \to b$ denotes a function which takes any type $a$ to any type $b$, and $f: a \to a$ denotes a function that takes any type $a$ and produces \emph{the same} type $a$.

The function operator $\to$ can be chained for a function to accept multiple parameters which we indicate as $f: a \to b \to c$. It's important to note that $\to$ is a \emph{binary} and \emph{right-associative} operator, meaning that

\[ f: a \to b \to c \quad \equiv \quad f: a \to (b \to c) \]

This effectively means that a function always takes exactly one argument and produces another function. In the example above, the returned value is a new function with type $b \to c$, which can then be applied to a value of type $b$ producing a final value of type $c$.

If we group the function operator differently, we can describe functions that take other functions as argument. Consider for example $f: (a \to b) \to c$, indicating that $f$ takes a function of type $a \to b$ and returns a value of type $c$. The possibility to apply functions to functions and produce functions as return values is known as \emph{higher-order} functions.

Closely related is the concept of functions as so-called \emph{first-class citizen}, which basically means that a function can be passed around and transformed just like any other type.

We consider all functions to be \emph{anonymous}, meaning that they do not have a name associated with them in their definition. They can then be bound to names using the \emph{bind} operation (syntactically denoted $=$). The following are examples of how we define concrete implementations of functions (later we will use Kite specific syntax, but for now we stick with simple mathematical notation).

\begin{align*}
id &= \fn{x}{x} & \text{The identify function: returns the argument, unchanged}\\
add &= \fn{ x, y }{ x + y }  & \text{Sum of two values}\\
max &= \fn{ x, y }{ \ite{x > y}{x}{y} }  & \text{Maximum of two values}\\
\end{align*}

Here we use the $\lambda$-notation for functions, using the ``maps to'' operator $\mapsto$ to indicate concrete implementation. We indicate function application using standard parenthesis, for instance $sum = add(2, 5)$.

A very powerful concept that arises naturally from that of higher-order functions, is that of partial application and currying. With partial application it is possible to apply a single argument to a function which takes multiple arguments, which produces a new function that accepts the remaining arguments. For instance, given the function $add$ from above, we can create a new function:
\begin{align*}
increment &: Int \to Int\\
increment &= add(1)
\end{align*}

$increment$ takes an $Int$ as argument and returns the argument, plus one. Currying is closely related, but is concerned with transforming a function of type $f : a \times b \to c$ to the function $f' : a \to b \to c$, thus enabling partial application. Here we used another notation, namely $\times$, which denotes the Cartesian product of two values, which we will further describe as a \emph{pair} of values.

As can be observed from the above, functional languages are closely related to lambda calculus, and can be seen as an extension. Common to them both is that almost everything is regarded as an expression, meaning that everything conveys a \emph{value}. For instance, in imperative languages the \code{if} construct is a statement, controlling flow of execution, whereas in a functional language, it expresses a choice between two values.

When working in an imperative language, mutation of variables is a core feature that is used in almost all parts of a program. In a functional language however, it is common that variables are not in fact variable in the sense of being mutable, but rather declarations of values. For instance, to calculate the sum of a list of numbers, consider the following pseudo-code:

\begin{pseudo}
// imperative style
function sum(nums) do
  var s = 0
  foreach num in nums do
    s = s + num
  end
  return s
end

// functional style
function sum(nums) do
  return if length of nums is 0
    then 0
    else head(nums) + sum(tail(nums))
end
\end{pseudo}

Here we can see how the imperative code continuously mutates the \code{s} variable, whereas the functional code instead leverages recursion (and also is an example of the previously mentioned \code{if} as an expression).

The idea of immutability is related to the strive for controlling side effects. In a purely functional language, a function will \emph{always} produce the same value, given the same input. This makes the code much easier to reason about, as one does not have to worry about what state a certain function is in at a given time. It can also help the compiler make optimizations, which we will discuss in more detail later. Pure languages are, however, not practical for writing useful programs because, for instance, IO is inherently not pure. You cannot know in advance what input the user will give you, and a truly pure function cannot generate random numbers, get the time, etc. Solutions have been developed to keep a language pure while still maintaining side effecting operations, such as IO. For instance, Haskell uses the concept of Monads to force side effecting functions to declare which effects they might have and provide fallback strategies in case of failure.


\subsection{Implementing a compiler}
Most modern compilers are built using more or less the same architecture. The source code of the compiled language is first analyzed lexically (a process sometimes referred to as \textbf{lexing}) into a stream of lexical tokens. This can be compared to splitting a natural language text into a list of words. This stream of tokens is then parsed into nodes in an Abstract Syntax Tree (AST), using a \textbf{parser}. This can again be compared to natural languages by imaging a list of words and punctuation being parsed into phrases that have ``meaning''.

The AST can have various forms and structures, but most commonly it is a representation of the source code in the form of data structures native to the language of implementation. The AST can be transformed for different purposes, for instance, to give correct precedence to operators.

Going further, it can be analyzed in many different ways, for example to perform \textbf{type checking} and ensure that no undefined identifiers are referenced. It is also possible to perform various optimizations on the AST. Dead (unused) code can be detected and removed, common patterns of code can be transformen into a more efficient structure, functions can be inlined to remove the overhead of making a function call and much more. This will be discussed in detail later.

Finally, the AST can be used to emit code for the target language of the compiler, a process known as \textbf{code generation}. It is worth noting that many different code generators can be implemented for the same AST, to enable targeting of multiple architectures and machines using the same compiler.

We are going to implement all of these modules, some more thoroughly than others, and in such a way that it can be easily extended and maintained.
